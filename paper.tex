\documentclass{IEEEtran}

\usepackage{graphicx}
\usepackage{color}
\usepackage[bookmarks=false]{hyperref}

\newcommand{\todo}[1]{\marginpar{\parbox{18mm}{\flushleft\tiny\color{red}\textbf{TODO}:
      #1}}}

\begin{document}

\title{Performance Evaluation of Big Data Management Strategies\\ for Neuroinformatics}

\author{
  \IEEEauthorblockN{
    Val\'erie Hayot-Sasson and 
    Tristan Glatard
  }\\
  \IEEEauthorblockA{
    Department of Computer Science and Software Engineering, Concordia University, Montreal, Qu\'ebec, Canada
  }
}

\maketitle

%
% Abstract
%
\begin{abstract}
  
\end{abstract}

%
% Section
%
\section{Introduction} % 1 page with abstract

% Big Data Engines have been developed mostly for text applications, 
%  in particular Web. Mention Spark and Dask

% Scientific applications, including neuroinformatics are different: CLI, binary data, shared HPC clusters (although ref to cloud).
% In neuroinformatics, nipype, psom, etc
% This follows the current trend to big data science, data sets *and derived data* are larger (give examples).

% Our question: is it relevant to use Big Data engines for neuroinformatics? 
% Implications: migrate existing pipelines, adopt a new framework (costly)
% Some papers said yes, for instance Ariel's and the Spark provenance one (https://github.com/UFFeScience/SAMbA).
% Perhaps this too: https://www.sciencedirect.com/science/article/pii/S0925231218311470 Spark is being used to implement neuroinformatics pipelines, but is it
% the right tool?
% But a quantitative evaluation is lacking.

% More precisely, we intend to quantify the effect of in-memory computing, data locality, lazy evaluation
% on various use cases. We intentionally leave scheduling out here.

% The use of a Big Data engine is not obvious when tmpfs and asynchronous writes
% are available and could be used to emulate in-memory computing. Our goal is to quantify when and how Big Data engines will be useful.
% Say that we don't use HDFS because we focus on clusters and HDFS on multi-tenant clusters is difficult (refer to hadoop on demand).

% Talk about pilot jobs and say we don't consider them here.

% "```Disk and network I/O, of course, play a part in Spark performance 
% as well, but neither Spark nor YARN currently do anything to actively 
% manage them.```"
% http://blog.cloudera.com/blog/2015/03/how-to-tune-your-apache-spark-jobs-part-2/


% Here we focus on performance but we are aware of other desirable properties of workflow engines, including provenacne.

% Outline

% Talk about containers somewhere.

\section{Materials and Methods} % 4 pages


\subsection{Engines} % 1.5 pages

\subsubsection{Apache Spark}

%% Main concepts
% detailed description relevant to the experiment
% how shuffle works
% how data locality is implemented
% spills to disk when dataset (and derived data) is too large for memory
% Lazy evaluation
% overlay cluster: not considered here as this would be scheduling

%% Technical details
% data serialization to Java (we use Python as most neuroinformatics do)
% limitation in size of RDD element, due to java implementation of binary data
% memory management: don't give too much memory to executors or GC
%  many places recommend limiting to 5 cores/executor, no one explains why
% more on tuning?

\subsubsection{Nipype}

% general description relevant to the experiment
% two plugins
% provenance
% memory management (bad?)

\subsubsection{Execution modes} % perhaps find a better title, it's not informative

% in-memory (Spark only)

% tmpfs: two extra in-memory copies. memfs could address that but we don't have it yet.
% Spark: when just passing filenames. turns out i need to set `preservePartitioning` to True
% local disk: write buffer issue

% shared fs: Lustre

% Make a table to explain where data locality and in-memory computing are
% available. 

% What happens when dataset and derived data is too large for disk?

\subsection{Simulation} % is it included? 0.5 page

\subsection{Models: Write Buffer} % perhaps other models if relevant. 0.5 page

% Perhaps refer to this: https://ieeexplore.ieee.org/abstract/document/5496998

% 15 cores per executor can lead to bad I/O throughput
% http://blog.cloudera.com/blog/2015/03/how-to-tune-your-apache-spark-jobs-part-2/

% measured disk bandwidth with random I/O, very different from sequential

\subsection{Experiments} % 1.5 pages

\subsubsection{Infrastructure} % 1/4 a page (half a col)

% Dedicated Compute nodes with SLURM
% Lustre server

\subsubsection{Datasets} % 1/4 of a page (half a col)

% BigBrain
% Some BIDS dataset for fmriprep

\subsubsection{Applications} % 1 page

% Incrementation (avoid cache effects in binarization)

% average (1 shuffle)
% Nipype: can't use SLURM plugin so had to actively poll.

% kmeans (many shuffles)

% Example BIDS app

% fmriprep?

% Explain how we tweaked scheduling in Nipype to make it similar to Spark, 
% otherwise we're only measuring scheduling differences. Needs a discussion on scheduling somewhere.


\section{Results} % 2 pages

% We checked that scheduling was similar in both cases (show a two Gantt charts to illustrate)

% interactive gantt charts?

\section{Discussion} % 2 pages

\subsection{General concepts}

% Shared fs vs local disk vs in-memory: what do we gain?

% Can we emulate in-memory computing using write buffers or tmpfs?

% Scheduling: load balance as much as you can.
% (which isn't what nipype multiproc does)

% tmpfs and local disks fill up: need cleanup.

% fault-tolerance: do we have anything to discuss here?

\subsection{Engine-specific observations}

\section{Conclusion} % 1 page with refs

% When is it useful to use a Big Data engine?

% Future work: scheduling in a shared environment.

\section{Acknowledgments}


\bibliographystyle{IEEEtran} 
\bibliography{IEEEfull,biblio}

\end{document}























